\section{Criterios para evaluar raíces}

\subsection{Regla de los signos de Descartes}

%%%%%%%%%%%%%%%%%%%%%
{Variaciones de signo}
Si un polinomio $P(x)$ tiene coeficientes reales, escritos sus \emph{exponentes en forma descendiente} y \emph{omitiendo exponentes con coeficiente cero}, entonces una \emph{variación de signo} ocurre siempre que dos signos opuestos. 


%%%%%%%%%%%%%%%%%%%%%
{}
\begin{problema}
\begin{itemize}
\item $x^{2}+4x+1$ tiene $0$ variaciones de signo.
\item $2x^{3}+x-6$ tiene $1$ variación de signo.
\item $x^{4}-3x^{2}-x+4$ tiene $2$ variaciones de signo.
\item $5x^{7}-3x^{5}-x^{4}+2x^{2}+x-3$ tiene  $3$ variaciones de signo.
\end{itemize}

\end{problema}


%%%%%%%%%%%%%%%%%%%%%
{Regla de los signos de Descartes}
\begin{proposicion}
Sea $P$ un polinomio con coeficientes reales
\begin{enumerate}
\item El número de \emph{ceros reales positivos} es o bien igual al número de variaciones de signo en $P(x)$ o bien menor este número por un número par. 
\item El número de \emph{ceros reales negativos} es o bien igual al número de variaciones de signo en $P(-x)$ o bien menor este número por un número par. 
\end{enumerate}

\end{proposicion}


%%%%%%%%%%%%%%%%%%%%%
{}
\begin{problema}
Use la regla de los signos de Descartes para estimar el número posible de ceros reales negativos y positivos del polinomio
\begin{align*}
P(x)= 3x^{6}+4x^{5}+3x^{3}-x-3
\end{align*}

\end{problema}

\begin{solucion}
	$P(x)= 3x^{6}+4x^{5}+3x^{3}-x-3$ tiene una única raíz real positiva y o bien tres o bien una raíces real negativas.
\end{solucion}


%%%%%%%%%%%%%%%%%%%%%% 
\subsection{Teorema de las Cotas}
%%%%%%%%%%%%%%%%%%%%%
{}
Diremos que $m\in \R$ es una \emph{cota inferior} y $M\in \R$ es una cota superior para el conjunto de \emph{ceros reales} de un polinomio si para cada raíz $c$ tenemos que 
\begin{align*}
	m \leq c \leq M.
\end{align*}


%%%%%%%%%%%%%%%%%%%%%
{}
\begin{teorema}
Sea $P(x)$ un polinomio con coeficientes reales.
\begin{enumerate}
\item Si se divide $P(x)$ entre $x-b$ con $b>0$ usando división sintética, y si la fila de coeficientes del cociente y residuo tiene entradas \emph{no negativas}, entonces $b$ es una cota superior para los ceros reales de $P(x)$.
\item Si se divide $P(x)$ entre $x-a$ con $a<0$ usando división sintética, y si la fila de coeficientes del cociente y residuo tiene entradas \emph{alternantemente no positivas y no negativas}, entonces $a$ es una cota inferior para los ceros reales de $P(x)$.
\end{enumerate}

\end{teorema}


%%%%%%%%%%%%%%%%%%%%%
{}
	\begin{problema}
		Muestre que todos los ceros reales del polinomio \begin{align*}
			P(x)=x^4-3x^{2}+2x-5
		\end{align*}
		están entre $-3$ y $2$.
	\end{problema}


%%%%%%%%%%%%%%%%%%%%%
{}
	\begin{problema}
		
		Factorice completamente el polinomio 
		$P(x)=2x^{5}+5x^{4}-8x^{3}-14x^{2}+6x+9$
	\end{problema}

\begin{solucion}
	$P(x)=\left( x-1 \right)\left( 2x-3 \right)\left( x+1 \right)^{2}\left( x+3 \right)$
\end{solucion}


