\section{Núcleo e imagen}


\begin{definicion}
 El \emph{núcleo} de una transformaci\'on lineal $T:V\to W,$ donde $V$ y $W$ son espacios vectoriales, es el
conjunto
$$
\ker(T)=\set{v\in V | T(v)=0}.
$$

La imágen de $T:V \to W$ es el conjunto
$$
\Im(T)=\set{w\in W | \exists v\in V, T(v)=w}.
$$
\end{definicion}

\begin{proposicion}
$\ker(T) < V, \Im(T)<W.$
\end{proposicion}

\begin{proof}
 Si $\a\in \R$ y $u,v \in \ker(T),$ entonces
 \begin{align*}
  T(\a u +v) &= \a T(u) + T(v) & \text{(Por linealidad de $T$)} \\
  &= \a 0 + 0 & \text{(Porque $T(u)=T(v)=0$)} \\
  &= 0 &
 \end{align*}
Por tanto, $\ker(T)<V.$

Ahora, si $\a\in \R$ y $w,w' \in \Im(T),$ entonces Existen $v,v'\in V$ tales que $T(v)=w, T(w')=v'.$ Por lo cual, 
\begin{align*}
 \a w + w' &= \a T(v) + T(v') \\
 &= T(\a v + v').
\end{align*}
Como $\a v + v' \in V,$ entonces $\a w + w' \in W.$
\end{proof}

\begin{problema}
\label{texmp:ker}
 Encontrar un conjunto de vectores que generen $\ker(T),$ para la tranformaci\'on lineal $T$ dada por
(\ref{trans_exmp}).
\end{problema}

\begin{solucion}
 Supongamos que
 $$
 T\begin{bmatrix}
    x\\y\\z
   \end{bmatrix}=
   \begin{bmatrix}
    x+2y \\ z-y\\ 2x+7y-3z
   \end{bmatrix}=
   \begin{bmatrix}
    0\\0\\0
   \end{bmatrix}
 $$
 Esto equivale a resolver el sistema de ecuacuones
\begin{align*}
 x+2y&=0\\
 z-y&=0 \\
 2x+7y-3x&=0,
\end{align*}
que podemos reescribir en forma matricial como
$$
\begin{bmatrix}
1 & 2 & 0 & 0 \\
0 & -1 & 1& 0 \\
2 & 7 & -3 & 0 
\end{bmatrix}
$$
y utilizando Gauss-Jordan, se reduce a
$$
\begin{bmatrix}
 1 & 2 & 0 & 0 \\
 0 & 1 & -1 & 0 \\
 0 & 0 & 0 & 0
\end{bmatrix},
$$
es decir, tenemos dos ecuaciones con tres incognitas
\begin{align*}
 x + 2y &= 0 \\
y -z &=0
 \end{align*}
por lo que sustituyendo $y=z=t,$ tenemos que
$$
\begin{bmatrix}
 x\\y\\z
\end{bmatrix}
=\begin{bmatrix}
  -2t \\ t \\ t
 \end{bmatrix}
 =t\begin{bmatrix}
   -2 \\ 1 \\ 1
  \end{bmatrix}
$$,
es decir, todos los vectores en $\ker(T)$ son multiplos de $$\begin{bmatrix}
   -2 \\ 1 \\ 1
  \end{bmatrix}.$$

  De manera equivalente,
  $$
\ker(T)=\gen{\begin{bmatrix}
   -2 \\ 1 \\ 1
  \end{bmatrix}}.
  $$
\end{solucion}

\begin{problema}
\label{texmp:im}
  Encontrar un conjunto de vectores que generen $\Im(T),$ para la tranformaci\'on lineal $T$ dada por
(\ref{trans_exmp}).
\end{problema}

\begin{solucion}
 Un vector en $\Im(T)$ es de la forma,
 $$
\begin{bmatrix}
 x+2y \\ z-y \\ 2x+7y-3z
\end{bmatrix}
= x \begin{bmatrix}
     1 \\ 0 \\ 2
    \end{bmatrix}
    + y \begin{bmatrix}
         2 \\ -1 \\ 7
        \end{bmatrix}
+ z\begin{bmatrix}
0 \\ 1 \\ -3
\end{bmatrix},
 $$
 por lo que $\Im(T)$ estar\'ia generado por los vectores
 $$
u=\begin{bmatrix}
     1 \\ 0 \\ 2
    \end{bmatrix},
    v=\begin{bmatrix}
         2 \\ -1 \\ 7
        \end{bmatrix},
        w=\begin{bmatrix}
0 \\ 1 \\ -3
\end{bmatrix}.
 $$

Sin embargo, por el ejercicio anterior, $w=2u-v,$ y por tanto
$$
\begin{bmatrix}
 x+2y \\ z-y \\ 2x+7y-3z
\end{bmatrix} = xu+yv+wz=(x+2z)u + (y-z)v.
$$

De hecho, para cualesquiera $\lambda,\mu$, si escogemos una soluci\'on de las ecuaciones las ecuaciones
\begin{align*}
 \lambda &= x+2z \\
 \mu &= y-z,
\end{align*}
podemos escribir
$$
\begin{bmatrix}
 x+2y \\ z-y \\ 2x+7y-3z
\end{bmatrix} = xu+yv+zw=\lambda u + \mu v.
$$
Es decir,
$$
\Im(T)=\gen{u,v}=\gen{\begin{bmatrix}
     1 \\ 0 \\ 2
    \end{bmatrix}, \begin{bmatrix}
         2 \\ -1 \\ 7
        \end{bmatrix}}.
$$

\end{solucion}

\subsection*{Ejemplos}

\begin{problema} Encuentre un conjunto de vectores, con el m\'inimo número de elementos posible, que generen $\ker(T)$ e
$\Im(T)$ para cada una de las transformaciones lineales del ejercicio \ref{exe:trans}.
\end{problema}