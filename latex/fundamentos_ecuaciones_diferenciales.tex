\section{Fundamentos de Ecuaciones Diferenciales}

\subsection{Conceptos básicos}

\begin{itemize}
	\item 	Una \emph{ecuación diferencial} es una ecuación que involucra derivadas o diferenciales de una o varias variables.
	\item Si la ecuación sólo involucra derivadas respecto a una única variable independiente, diremos que es \emph{ordinaria}.  En otro caso, que es \emph{parcial}.
	\item Si la ecuación involucra derivadas de orden $n$, pero no de orden más alto, diremos que la propia ecuación es de \emph{orden $n$.}
\end{itemize}

%%%%%%%
\begin{resuelto}

\begin{enumerate}
	\item
		\label{exmp 02_01}
		\begin{align*}
		\left(y''\right)^{2}+3x=2\left(y'\right)^3
		\end{align*}
	\item
		\label{exmp 02:02}
		\begin{align*}
		\dfrac{dy}{dx}+\dfrac{y}{x} = y^2
		\end{align*}
	\item
	\label{exmp 02:03}
		\begin{align*}
			\dfrac{d^{2}Q}{dt^{2}}-3\dfrac{dQ}{dt}+2Q = 4\sin(2t)
		\end{align*}
	\item
		\label{exmp 02:04}
		\begin{align*}
			\dfrac{dy}{dx}=\dfrac{x+y}{x-y}
		\end{align*}

		De manera equivalente
		\begin{align*}
			(x+y)dx+(y-x)dy = 0
		\end{align*}
	\item
		\begin{align*}
			\dfrac{\partial^{2}V}{\partial x^{2}} +
			\dfrac{\partial^{2}V}{\partial y^{2}} = 0
		\end{align*}
\end{enumerate}

\end{resuelto}
%%%%%%%
\subsection{Constantes arbitrarias}

	Una constante arbitraria es un valor que es independiente de las variables involucradas en la ecuación.

	Generalmente las denotaremos con las primeras letras del alfabeto:
	\begin{align*}
		A,B,C,c_{1},c_{2},...
	\end{align*}

%%%%%%%%%%%

	\begin{ejemplo}
		En la ecuación
		\begin{align*}
		y = x^{2}+c_{1}x+c_{2}
		\end{align*}
		los símbolos $ c_{1}, c_{2} $ representan constantes arbitrarias.

	\end{ejemplo}

%%%%%%%

	\begin{ejemplo}
		La relación $ y = Ae^{-4x+B} $ se puede reescribir como $ y = Ce^{-4x} $.  Por lo que sólo involucra una constante arbitraria.
	\end{ejemplo}

%%%%%%%
\begin{observacion}

	Siempre reduciremos las ecuaciones al número mínimo de constantes arbitrarias, a las que llamaremos \emph{esenciales}.

\end{observacion}
%%%%%%%%%%
\subsection{Soluciones de ecuaciones diferenciales}

\begin{itemize}
	\item 	Una \emph{solución de una ecuación diferencial} es una relación entre las variables que está libre de derivadas, y que satisface la ecuación diferencial en al menos un intervalo.
	\item
	Una \emph{solución general} de una ecuación diferencial de orden $ n $ es aquella que involucra $ n $ constantes arbitrarias esenciales.
	\item 	Una \emph{solución particular} es aquella que se obtiene de una general, sustituyendo valores específicos en las constantes arbitrarias.
	\item 	Una \emph{solución singular} es una aquella que no se puede obtener de la solución general sólo especificando valores para las constantes arbitrarias.
\end{itemize}

%%%%%%%

	\begin{resuelto}
		\label{exmp 02:06}
		Demuestra que $ y=x^{2}+c_{1}x+c_{2} $ es una solución general de $ y''=2 $.
	\end{resuelto}

	\begin{resuelto}
		\label{exmp 02:08}
		Verifica que $ y = x^2-3x+2 $ es una solución particular de $ y''=2 $.
	\end{resuelto}

	\begin{ejemplo}
		La solución general de $ y = xy'-y'^{2} $ es $ y = cx-c^{2} $.

		 Sin embargo, $ y=\dfrac{x^{2}}{4} $ es una solución que no se puede obtener sustituyendo $ c $.  Por tanto, es una solución particular.
	\end{ejemplo}


\begin{definicion}
	Una solución general de orden $ n $ tiene $ n $ parámetros (constantes arbitrarias esenciales) y por tanto, geométricamente representa una \emph{familia de curvas $n-$paramétrica. }

	De manera reciproca, una relación con $ n $ constantes arbitrarias (también llamada \emph{primitiva}) tiene asociada una ecuación diferencial de orden $n$ (de la cual es solución general), llamada la \emph{ecuación diferencial de la familia}.

\end{definicion}
%%%%%%%

