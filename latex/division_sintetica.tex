%301
\section{M\'etodo de Horner y Divisi\'on Sint\'etica}


	\begin{problema}
		Consideremos evaluar el siguiente polinomio
		$$
		p(x)=6x^{2}+3x-2
		$$ en $x=9.$
	\end{problema}
	



	\begin{align*}
		p(9)&=6(9)^{2}+3(9)-2  \\
		&=6(81)+3(9)-2  \\
		&=486+27-2  \\
		&=513-2=511
	\end{align*}
	



	Consideraremos una forma alternativa de evaluar, conocida como \emph{m\'etodo de Horner.}



	Primero, reescribimos el polinomio de la siguiente manera 
	\begin{align*}
		p(x)&=\left( \textcolor{blue}{6}x^{2} + \textcolor{green}{3}x \right) \textcolor{red}{-2}\\
		&= \left( \textcolor{blue}{6}x+\textcolor{green}{3} \right)x\textcolor{red}{-2} \\
		&= \left( (\textcolor{blue}{6}) x + \textcolor{green}{3} \right) x \textcolor{red}{-2}
	\end{align*}
	



	Al evaluar, realizamos las siguientes operaciones
	\begin{align*}
		p(9)&=\left( (\textcolor{blue}{6}) 9  \textcolor{green}{+3} \right) 9 \textcolor{red}{-2}\\
		&=\left( \textcolor{blue}{54} \textcolor{green}{+3} \right)9\textcolor{red}{-2} \\
		&= \left( \textcolor{green}{57} \right)9\textcolor{red}{-2}\\
		&=\textcolor{green}{513}\textcolor{red}{-2}\\
		&=\textcolor{red}{511}
	\end{align*}
	



	\begin{observacion}
		Aunque con el m\'etodo anterior, hemos realizado algunos pasos más, hemos evitado el uso de \emph{exponentes}.  Ahora, todo se reduce a \emph{multiplicaciones y sumas.}
	\end{observacion}
	



	El m\'etodo anterior se puede sintetizar de la siguiente manera
	\begin{center}
		\begin{tabular}{l|lll}
			9 & \textcolor{blue}{6} & \textcolor{green}{+3} & \textcolor{red}{-2}\\
			& $\downarrow$ & 54 & 513\\\hline
			& 6 & 57 & 511
		\end{tabular}
	\end{center}
	



	De manera general, 
	\begin{center}
		\begin{tabular}{l|lll}
			x & \textcolor{blue}{6} & \textcolor{green}{+3} & \textcolor{red}{-2}\\
			& $\downarrow$ & $6x$ & $(6x+3)x$\\\hline
			& $6$ & $6x+3$ & $\left( 6x+3 \right)x-2$
		\end{tabular}
	\end{center}
	



	\begin{observacion}
		La última expresi\'on $\left( 6x+3 \right)x-2$ es igual a nuestro polinomio
		$$
		6x^{2}+3x-2.
		$$
		
		
		
		Al procedimiento anterior se le conoce como \emph{divisi\'on sint\'etica.}
	\end{observacion}
	



	\begin{problema}
		Evalue $p(x)=2x^{3}-7x^{2}+5$ en $x=3$ utilizando
		\begin{enumerate}
			\item evaluaci\'on directa 
			\item el m\'etodo de Horner 
			\item divisi\'on sint\'etica
		\end{enumerate}
		
		
	\end{problema}
	



	\begin{problema}
		Evalue $p(x)=3x^{5}+5x^{4}-4x^{3}+7x+3$ en $x=-2$ utilizando
		\begin{enumerate}
			\item evaluaci\'on directa 
			\item el m\'etodo de Horner 
			\item divisi\'on sint\'etica
		\end{enumerate} 
	\end{problema}



	\begin{problema}
		Evalue $p(x)=x^{3}-7x+6$ en $x=1$ utilizando
		\begin{enumerate}
			\item evaluaci\'on directa 
			\item el m\'etodo de Horner 
			\item divisi\'on sint\'etica
		\end{enumerate} 
	\end{problema}



	\begin{definicion}
		Si al evaluar un polinomio $p(x)$ en $x=c,$ obtenemos 
		$$
		{\color{red} p(c)=0},
		$$
		diremso que $c$ es un \emph{ra\'iz} o \emph{``cero''} del polinomio $p(x).$
	\end{definicion}



	\begin{problema}
		Evalue $p(x)=x^{4}-3x^{3}-13x^{2}+15x$ en $x=-3,0,1,5$ utilizando
		\begin{enumerate}
			\item evaluaci\'on directa 
			\item el m\'etodo de Horner 
			\item divisi\'on sint\'etica
		\end{enumerate} 
	\end{problema} y compruebe que son \emph{ceros} del polinomio.

