\section{Transformaciones lineales}

\subsection*{Definici\'on y ejemplos}
\begin{definicion} Sean $V,W$ espacios vectoriales. Decimos que $T:V \to W$ es una \emph{transformaci\'on lineal} si para
todos $\a \in \R, u,v \in V$ se cumple
\begin{enumerate}
\item $T(u+v)=T(u)+T(v)$ ($T$ abre sumas)
 \item $T(\a u)=\a T(u)$ ($T$ saca escalares)
\end{enumerate}
o de manera equivalente
$$
T(\a u +v)=\a T(u) + T(v),
$$
es decir, \emph{$T$ respeta la estructura de espacio vectorial.}

En el caso $V=W,$ decimos que $T:V\to V$ es un operador y al conjunto de operadores en $V$ lo denotamos por $L(V).$
En el caso $W=\R,$ decimos que $T:V\to \R$ es un funcional en $V.$
\end{definicion}

\begin{resuelto}
 Sea $a=[a_{1},...,a_{n}]\in \R^{n}$ fijo y definamos $T:\R^{n}\to \R, \, T(u)=a \cdot u.$ Entonces, $T$ es una
transformaci\'on lineal.
\end{resuelto}


\begin{resuelto}
\label{matriz:trans_lin}
Sea $A=\begin{bmatrix}
        a_{ij}
       \end{bmatrix}
$ una matriz de dimensi\'on $n \times m,$ donde $n$ indica el número  de columnas y $m$ el de renglones.

Si definimos $T:\R^{n}\to \R^{m}, \, T(u)=Au,$ entonces $T$ es una tranformaci\'on lineal. En otras palabras, cada
matriz define una transformaci\'on lineal. Lo inverso tambi\'en es cierto.

Sea $$e_{k}=\begin{bmatrix}
             0\\
             \vdots \\
             1 \\
             \vdots \\
             0
            \end{bmatrix}
$$
el vector columna con un $1$ en la $k-$\'esima posici\'on y ceros en el resto, y sea $T:\R^{n}\to \R^{m}$ una
transformaci\'on lineal. Entonces, $T(e_{k})\in \R^{m}$ y digamos que es de la forma
$$
T(e_{k})=\begin{bmatrix}
          a_{1k}\\
          \vdots \\
          a_{mk} \\
         \end{bmatrix}.
$$

Si definimos $A=[a_{ij}]\in M_{mn}$ entonces $T(e_{k})=Ae_{k}, \, k=1,...,n.$ Por linealidad tanto de $T$ como de $A,$
obtenemos que $T(x)=Ax,$ para todo vector $x\in \R^{n}.$
\end{resuelto}

\begin{resuelto}
Las siguientes transformaciones son lineales:
\begin{itemize}
 \item $T:C^{1}[0,1]\to C^{0}[0,1],$
 $$
T(f)(x)=f'(x).
 $$
 \item $T:C^{0}[0,1]\to \R,$
$$
T(f)=\int_{0}^{1}f(t)dt.
$$
\item $T:C^{0}[0,1]\to C^{1}[0,1],$
$$
T(f)(x)=C+\int_{0}^{x}f(t)dt,
$$
donde $x\in[0,1]$ y $C\in \R$ es alguna constante.
\end{itemize}

\end{resuelto}

\begin{resuelto}
 Indique si la siguiente transformaci\'on $T:\R^{3}\to \R^{3}$ es lineal y de ser as\'i, encuentre su representaci\'on
matricial.
\[
\label{trans_exmp}
 T\begin{bmatrix}
    x\\y\\z
   \end{bmatrix}=
   \begin{bmatrix}
    x+2y \\ z-y\\ 2x+7y-3z
   \end{bmatrix}
\]

\end{resuelto}

\begin{solucion}
 La prueba de que la transformaci\'on es lineal se deja al lector. Ahora bien,
 $$
T\begin{bmatrix}
  1\\0\\0
 \end{bmatrix}
=\begin{bmatrix}
  1\\0\\2
 \end{bmatrix},
 T\begin{bmatrix}
  0\\1\\0
 \end{bmatrix}
=\begin{bmatrix}
  2\\-1\\7
 \end{bmatrix},
T\begin{bmatrix}
  0\\0\\1
 \end{bmatrix}
=\begin{bmatrix}
  0\\1\\-3
 \end{bmatrix}.
 $$

 Por lo tanto, la representaci\'on matricial de $T$ esta dada por
 $$
=\begin{bmatrix}
1 & 2 & 0 \\
0 & -1 & 1 \\
2 & 7 & -3
 \end{bmatrix}
 $$
\end{solucion}


\subsection*{Operadores en  $\R^{n}$}

Sean $T,S \in L(\R^{n}).$ La composici\'on $TS,$ es decir, $TS(x)=T(S(x))$ es de nuevo un operador y de hecho, si
$B=[b_{ij}]$ es la matriz asociada a $T$ como en el ejemplo \ref{matriz:trans_lin} y $A=[A_{ij}]$ la asociada a $S,$
entonces la matriz asociada a $TS$ es $C=[c_{ij}]$ conjunto $$
c_{ij}=\sum_{k}^{n}b_{ik}a_{kj}.
$$
Decimos que $C=BA$ es el producto de de $B$ con $A$ (es este orden), y esta composicion es asociativa.

El operador de $T$ con $S$ suma esta definido como $(T+S)(u)=T(u)+S(u),$ y de hecho tiene asociada la matriz $$
\begin{bmatrix}
 b_{ij}+a_{ij}
\end{bmatrix}.
$$

Dos operadores especiales en $R^{n}$ son la \emph{transformaci\'on cero} $0(x)=0$ y la \emph{identidad} $\id(x)=x.$
\begin{resuelto}[\dag]
 Encuentre la matriz asociada a los operadores cero e identidad.
\end{resuelto}

Podemos definir la multiplicaci\'on de operadores por escalares de la siguiente forma. $(\a T)(u)=\a T(u).$ De esta
manera, con la operaci\'on suma entre operadores y esta multiplicaci\'on por escalares, resulta que $L(\R^{n})$ es un
espacio vectorial.

Finalmente, si para $P\in L(\R^{n})$ existe $Q\in L(\R^{n}),$ de manera que $PQ=\id,$ decimos que $P$ es
\emph{invertible} y que $Q$ es el operador inverso de $P.$ Tambi\'en podemos escribir $Q$ como $P^{-1}.$
De hecho, si $A$ es la matriz asociada a $P,$ entonces $A^{-1}$ es la asociada a $P^{-1}.$




\subsection*{Ejemplos}

\begin{resuelto} \label{exe:trans}Verificar que las siguientes transformaciones son lineales, y encontrar la representaci\'on
matricial de cada una.
\begin{enumerate}
 \item (Proyecci\'on sobre el plano)$$
T\begin{bmatrix}
  x \\ y \\z
 \end{bmatrix}
=\begin{bmatrix}
  x \\ y \\0
 \end{bmatrix}
 $$

 \item $$
T\begin{bmatrix}
  x \\ y \\z
 \end{bmatrix}
 =\begin{bmatrix}
  x - y\\ y+z \\ 2x - y - z \\ -x+y+2z
 \end{bmatrix}
 $$

 \item $$
T\begin{bmatrix}
  x \\ y \\z
 \end{bmatrix}
 =\begin{bmatrix}
  2x - y +3z\\ 4x-2y+6z \\ -6x +3 y - 9z
 \end{bmatrix}
 $$

\item$$
T\begin{bmatrix}
  x \\ y
 \end{bmatrix}
 =\begin{bmatrix}
   2x-y\\ 2y-4x
  \end{bmatrix}
$$

\item$$
T\begin{bmatrix}
  x \\ y
 \end{bmatrix}
 =\begin{bmatrix}
   2x \\ 3x-y
  \end{bmatrix}
$$

\item$$
T\begin{bmatrix}
  x \\ y \\z
 \end{bmatrix}
 =\begin{bmatrix}
   2x-y+3z \\ 6x -3y +9z
  \end{bmatrix}
$$

\item $$
T\begin{bmatrix}
  x \\ y
 \end{bmatrix}
 =\begin{bmatrix}
2x+y \\ 2y-x \\ x+8y
  \end{bmatrix}
$$
\item
$$
T\begin{bmatrix}
  x\\y\\z
 \end{bmatrix}
=\begin{bmatrix}
  1x  -3z \\
   -y + 5z
 \end{bmatrix}
$$
\item
$$
T\begin{bmatrix}
  x \\ y
 \end{bmatrix}
=\begin{bmatrix}
  2x+y\\
  -4x + 2y \\
  8x + 4y
 \end{bmatrix}
$$
\item
$$
T=\begin{bmatrix}
   x\\y\\z
  \end{bmatrix}
=\begin{bmatrix}
  3x  -z \\
  -y + 2z \\
  15x  -2y  -z
 \end{bmatrix}
$$

%  \item(Rotaci\'on en el plano)$$
% T \begin{bmatrix}
%    r\cos(t) \\
%    r\sin(t)
%   \end{bmatrix}
% = \begin{bmatrix}
%    r\cos(t+a) \\
%    r\sin(t+a)
%   \end{bmatrix}
%  $$

%  \item $T:P_{2} \to P_{3}, (Tp)(x)=xp(x).$
%
%  \item $T:P_{3} \to P_{2}, T(a+bx+cx^{2}+dx^{3})=b+cx^{2}.$
\end{enumerate}
\end{resuelto}

\begin{resuelto}
 Encuentre una expresi\'on matemática para la transformaci\'on que rota un vector en el plano, con un ángulo $\phi$
en el sentido positivo (contrario a las manecillas del reloj). Indique si esta transformaci\'on es lineal y de serlo,
encuentre su representaci\'on matricial. \emph{Sugerencia: Exprese el vector en coordenadas polares.}
\end{resuelto}
