\section{Valores propios}


Como hemos visto, hacer cálculos que involucren matrices, por ejemplo multiplicar una matriz por un vector, puede ser
complicados por la cantidad de operaciones involucradas. En cambio, multiplicar por escalares es muy sencillo.
¿Podríamos encontrar alguna manera de convertir las operanciones con matrices en operaciones con escalares? En este
capítulo trataremos de responder esta pregunta.

\begin{definicion}
 Sea $T:\R^{n}\to \R^{n}$ una transformación lineal y $A$ su representación matricial en la base estandar. Si $\lam\in
\R$ y $v\in \R, \, v\neq 0$ tales que
$$
Av=\lam v,
$$
decimos que $\lam$ es un valor propio y $v$ un $\lam$-vector propio.
\end{definicion}

Supongamos que $F=\left( v_{1},...,v_{n} \right)$ es una base ordenada de vectores propios de $\R^{n},$ es decir,
$$
Av_{k}=\lam_{k}v_{k}, \, k=1,...n.
$$

Entonces, la representación matricial de $T:\R^{n}\to \R^{n}$ en la base $F$ es
$$
B=\begin{bmatrix}
 \lam_{1} & 0 & \dots & \dots & \dots & 0 \\
 0 & \lam_{2} & \dots & \dots & \dots & 0 \\
 \vdots & \vdots & \ddots & \dots & \dots & 0 \\
 0 & 0 & \dots & \lam_{k} & \dots & 0 \\
 \vdots & \vdots &\vdots &\vdots &\vdots &\vdots \\
 0 & \dots & 0 & \dots & 0 & \lam_{n}
\end{bmatrix}
$$
es decir, una matriz con los valores propios en la diagonal y ceros en otras partes.

Si expresamos un vector $v\in \R^{n}$ en esta base, tendría la forma
$$
v=c_{1}v_{1}+...+c_{n}v_{n},
$$
y aplicando la transformación, o de manera equivalente, multiplicando por $B$, obtendriamos
$$
T(v)=c_{1}\lam_{1}v_{1}+...+c_{n}\lam_{n}v_{n},
$$
es decir, simplemente haríamos operaciones con escalares. Por esta razón, es importante estudiar los valores y vectores
propios asociados a operadores en $\R^{n}$, es decir, transformaciones lineales $T:\R^{n}\to \R^{n}.$ Esta teoría se
conoce como \emph{espectral.}

\section{Valores propios}

 El primer paso para desarrollar la teoría espectral de un operador es determinar sus valores
propios. Antes, recordemos el siguiente criterio para determinar si un operador es invertible.

\begin{proposicion}
 Sea $T:\R^{n}\to \R^{n}$ una transformación lineal, y $A$ una representación lineal en alguna base de $\R^{n}$. Las
siguientes proposiciones son equivalentes:
\begin{enumerate}
 \item $A$ es invertible,
 \item $Av=0$ si y solo si $v=0,$
 \item $\det(A)\neq 0$.
\end{enumerate}
\end{proposicion}

La misma proposición se puede reescribir de la siguiente manera.

\begin{proposicion}
 Sea $T:\R^{n}\to \R^{n}$ una transformación lineal, y $M$ una representación lineal en alguna base de $\R^{n}$. Las
siguientes proposiciones son equivalentes:
\begin{enumerate}
 \item $M$ no es invertible,
 \item Existe un vector $v\neq 0,$ tal que $Mv=0,$
 \item $\det(A)= 0$.
\end{enumerate}
\end{proposicion}



Supongamos que $\lam$ es un valor propio de $A$ y $v$ un $\lam$-vector propio. Como $v = Iv,$ entonces
$$
Av=\lam v \ssi Av=\lam Iv \ssi (A-\lam I)v=0.
$$
Es decir, $v\in \ker(T)$ aunque $v\neq 0.$ Esto quiere decir que $A-\lam I$ no es invertible y por tanto,
$$
\det(A-\lam I)=0.
$$

Este es el criterio que buscabamos para localizar los valores propios.

\begin{definicion}
 Si $A\in M_{n\times n},$ entonces
 $$
p(\lam)=(-1)^{n}\det(A-\lam I)=\det(\lam I - A)
 $$
 se conoce como \emph{polinomio característico} de $A.$
\end{definicion}

\begin{observacion}
 $\lam$ es valor propio de $A$ si y solo si es raíz de $p(\lam).$
\end{observacion}

\begin{problema}
 Encuentre los valores propios, de la transformación lineal con representación matricial
\[
 \label{no:diagonal}
 A=\begin{bmatrix}
3 & 1 & -1 \\
2 & 2 & -1 \\
2 & 2 & 0
  \end{bmatrix}.
\]

\end{problema}

\begin{solucion}
 Primero determinamos el polinomio característico:
\begin{align*}
 p(\lam)&= -
\begin{vmatrix}
3 - \lam & 1 & -1 \\
2 & 2 - \lam & -1 \\
2 & 2 & -\lam
  \end{vmatrix}\\
  &=\lam^{3}-5\lam^{2}+8\lam-4\\ &=(\lam-1)(\lam-2)^{2}.
\end{align*}

Los valores propios de $A$ son las raices de $p(\lam)=(x-1)(x-2)^{2},$ es decir,
$$
\lam_{1}=1, \lam_{2}=2.
$$

Podemos verificar nuestra respuesta en \texttt{WxMaxima}, de la siguiente manera:

Primero, introducimos la matriz.

%\begin{center}
\noindent
%%%%%%%%%%%%%%%
%%% INPUT:
\begin{minipage}{8ex}{\color{red}\bf
\begin{verbatim}
(%i1)
\end{verbatim}}
\end{minipage}
\begin{minipage}{\textwidth}{\color{blue}
\begin{verbatim}
matrix(
 [3,1,-1],
 [2,2,-1],
 [2,2,0]
);
\end{verbatim}}
\end{minipage}
%%% OUTPUT:
\definecolor{labelcolor}{RGB}{100,0,0}
\begin{math}\displaystyle
\parbox{8ex}{\color{labelcolor}(\%o1) }
\begin{pmatrix}3 & 1 & -1\cr 2 & 2 & -1\cr 2 & 2 & 0\end{pmatrix}
\end{math}
%%%%%%%%%%%%%%%
%\end{center}

Posteriormente, calculamos el polinomio característico. En este caso, \texttt{WxMaxima} usará la definición $$
p(x)=\det\left( A-xI \right).
$$
%\begin{center}
\noindent
%%%%%%%%%%%%%%%
%%% INPUT:
\begin{minipage}{8ex}{\color{red}\bf
\begin{verbatim}
(%i2)
\end{verbatim}}
\end{minipage}
\begin{minipage}{\textwidth}{\color{blue}
\begin{verbatim}
charpoly(%, x), expand;
\end{verbatim}}
\end{minipage}
%%% OUTPUT:
\definecolor{labelcolor}{RGB}{100,0,0}
\begin{math}\displaystyle
\parbox{8ex}{\color{labelcolor}(\%o2) }
-{x}^{3}+5\,{x}^{2}-8\,x+4
\end{math}
%%%%%%%%%%%%%%%
%\end{center}

Finalmente, factorizamos el polinomio.
%\begin{center}
\noindent
%%%%%%%%%%%%%%%
%%% INPUT:
\begin{minipage}{8ex}{\color{red}\bf
\begin{verbatim}
(%i8)
\end{verbatim}}
\end{minipage}
\begin{minipage}{\textwidth}{\color{blue}
\begin{verbatim}
factor(%o2);
\end{verbatim}}
\end{minipage}
%%% OUTPUT:
\definecolor{labelcolor}{RGB}{100,0,0}
\begin{math}\displaystyle
\parbox{8ex}{\color{labelcolor}(\%o8) }
-{\left( x-2\right) }^{2}\,\left( x-1\right)
\end{math}
%%%%%%%%%%%%%%%
%\end{center}

Otra manera, más directa, es encontrar directamenta las raices del polinomio
%\begin{center}

\noindent
%%%%%%%%%%%%%%%
%%% INPUT:
\begin{minipage}{8ex}{\color{red}\bf
\begin{verbatim}
(%i13)
\end{verbatim}}
\end{minipage}
\begin{minipage}{\textwidth}{\color{blue}
\begin{verbatim}
realroots(%o2);
\end{verbatim}}
\end{minipage}
%%% OUTPUT:
\definecolor{labelcolor}{RGB}{100,0,0}
\begin{math}\displaystyle
\parbox{8ex}{\color{labelcolor}(\%o13) }
[x=2,x=1]
\end{math}
%%%%%%%%%%%%%%%

%\end{center}

Otra manera de obtener los valores propios es la siguiente:

\noindent
%%%%%%%%%%%%%%%
%%% INPUT:
\begin{minipage}{8ex}{\color{red}\bf
\begin{verbatim}
(%i21)
\end{verbatim}}
\end{minipage}
\begin{minipage}{\textwidth}{\color{blue}
\begin{verbatim}
eigenvalues(A);
\end{verbatim}}
\end{minipage}
%%% OUTPUT:
\definecolor{labelcolor}{RGB}{100,0,0}
\begin{math}\displaystyle
\parbox{8ex}{\color{labelcolor}(\%o21) }
[[1,2],[1,2]]
\end{math}
%%%%%%%%%%%%%%%
En este caso, el primer arreglo nos dice los valores propios, mientras que el segundo, nos dice sus
\emph{multiplicidades algebráicas,} que es el exponente que tienen asociado en el polinomio característico.
\end{solucion}

\subsection*{Ejemplos}

%Los ejercicios de esta sección se pueden encontrar en \cite[sec. 6.3]{G}.

\begin{problema}
\label{exe:diagonal}
 Encuentre los valores propios de las siguientes matrices. Verifique sus resultados usando \texttt{WxMaxima}.
 \begin{enumerate}

 \item $$
A=\bm{-2&-2\\-5&1}
 $$

 \item$$A=\bm{2&-1\\5&-2}$$

 \item$$A=\bm{3&2\\-5&1}$$
 
 \item $$A=\begin{bmatrix}
            4 & 2 \\
            3 & 3
           \end{bmatrix}
$$

\item$$
A=\begin{bmatrix}
   -6 & -3 & -25 \\
   2 & 1 & 8 \\
   2 & 2 & 7
  \end{bmatrix}
$$

\item $$
A=\begin{bmatrix}
   1 & -1 & 4 \\
   3 & 2 & -1 \\
   2 & 1 & -1
  \end{bmatrix}
$$
\item $$
A=\begin{bmatrix}
   3 & 2 & 4 \\
   2 & 0 & 2 \\
   4 & 2 & 3
  \end{bmatrix}
$$

\item$$A=\bm{1&1&-2\\-1&2&1\\0&1&-1}$$

\item$$A=\bm{7&-2&-4\\3&0&-2\\6&-2&-3}$$

\item$$A=\bm{-3&-7&-5\\2&4&3\\1&2&2}$$
\end{enumerate}
\end{problema}



