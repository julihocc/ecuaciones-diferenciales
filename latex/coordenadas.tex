\section{Coordenadas y cambios de base.}
\label{sec:coordenadas}

\subsection*{Coordenadas}

Si $E=\set{e_{1},...,e_{n}}$ es base de un espacio vectorial $V,$ entonces todo vector $v\in V$ se puede escribir de la
forma
\[
 \label{coordenadas}
 v=v_{1}e_{1}+...+v_{n}e_{n}.
\]

Esto es cierto para cualquier otro conjunto que genere $V.$ Lo importante de una base es que, debido a la independencia
lineal de $E$, esta manera de escribir el vector es \emph{única}.

Supongamos que podemos escribir $v=c_{1}e_{1}+...+c_{n}e_{n}.$ Entonces
\begin{align*}
 0&=v-v \\
 &=(v_{1}e_{1}+...+v_{n}e_{n})-(c_{1}e_{1}+...+c_{n}e_{n}) \\
&=(v_{1}-c_{1})e_{1}+...+(v_{n}-c_{n})e_{n}.
\end{align*}

Como $E$ es linealmente independiente, entonces $v_{1}-c_{1}=...=v_{n}-c_{n}=0.$ Es decir,
$$
v_{1}=c_{1},  ...,v_{n}=c_{n}.
$$
En otras palabras, los escalares $v_{1},...,v_{n}$ en la expresi\'on (\ref{coordenadas}) es \emph{única.}

Para simplificar la expresi\'on (\ref{coordenadas}) necesitamos el concepto de orden de una base.

\begin{definicion}
 Una base ordenada $\left( e_{1},...,e_{n} \right)$ es una sucesi\'on de vectores en $V$ tal que
 $\set{e_{1},...,e_{n}}$ es una base.

 Dos bases ordendas $\left( e_{1},...,e_{n} \right), \left( f_{1},...,f_{n} \right)$ son iguales si y solo si
$$e_{1}=f_{1},..., e_{n}=f_{n}.$$
\end{definicion}

\begin{observacion}
 Si intercambiamos un par de elementos de una base ordenada obtendremos una base ordenada distinta, aunque como
conjuntos sean diferentes.
\end{observacion}

\begin{resuelto}
 $$
\left( \begin{bmatrix}
        1 \\ 0
       \end{bmatrix},
       \begin{bmatrix}
        0 \\ 1
       \end{bmatrix}
 \right)
 $$ y
 $$
\left( \begin{bmatrix}
        0 \\ 1
       \end{bmatrix},
       \begin{bmatrix}
        1 \\ 0
       \end{bmatrix}
 \right)
 $$
 son dos bases ordenadas distintas de $\R^{2}.$
\end{resuelto}

Si consideramos $E$ como la base ordenada $\left( e_{1},...,e_{n} \right)$ entonces, la expresi\'on (\ref{coordenadas})
se puede escribir como
$$
\begin{bmatrix}
 v_{1} \\ ... \\ v_{n}
\end{bmatrix}_{E}.
$$
Decimos que $v_{1},...,v_{n}$ son las cordenadas de $v$ en la base $E.$

\begin{resuelto}
 Si consideramos la base ordenada $$E=(\bm{1\\1}, \bm{0\\1},)$$ de $\R^{2},$ entonces
 $$
\bm{u\\v}=\bm{u\\v}_{E}.
 $$

 En cambio, si consideramos $$
F=\left( \bm{0\\1}, \bm{1\\0} \right),
 $$
 entonces
 $$
\bm{u\\v}=\bm{v\\u}_{F}.
 $$
\end{resuelto}

\begin{definicion}
 La base $$
\left( \bm{1\\0\\ \vdots \\ 0}, \bm{0\\1\\ \vdots \\ 0}, ..., \bm{0\\0\\ \vdots \\ 1} \right)
 $$
 de $\R^{n}$ se conoce como \emph{base can\'onica.}
\end{definicion}

\subsection*{Cambios de base}

Supongamos que tenemos dos bases $B=\basis{e_{1},...,e_{n}}$ y $F=\basis{f_{1},..,f_{n}}$ de $\R^{n}.$ ¿Como podemos
comparar las cordenadas de un vector $v\in \R^{n}$ en ambas bases? Digamos que sus coordenadas son
$$
v=\bm{v_{1}\\ \vdots \\ v_{n}}_{B} = \bm{w_{1} \\ \vdots \\ w_{n}}_{F}.
$$

Para realizar la comparación, digamos que
las coordenadas de cada elemento de la base $F$ en la base $B$ son
$$
f_{k}=\bm{f_{k,1}\\ \vdots \\ f_{k,n}}_{B}.
$$

Ahora bien
\begin{align*}
 \bm{w_{1} \\ \vdots \\ w_{n}}_{F}
 &=w_{1}f_{1}+...+w_{n}f_{n} \\
 &=w_{1}\bm{f_{1,1}\\ \vdots \\ f_{1,n}}_{B}+...+w_{n}\bm{f_{n,1}\\ \vdots \\ f_{n,n}}_{B} \\
 &=\bm{w_{1}f_{1,1}+...+w_{n}f_{n,1}\\  \vdots \\
w_{1}f_{1,n}+...+w_{n}f_{n,n} }_{B},
\end{align*}
y como las coordenadas en una base son únicas, tenemos que
\begin{align*}
 \bm{v_{1}\\ \vdots \\ v_{n}} & = \bm{w_{1}f_{1,1}+...+w_{n}f_{n,1}\\  \vdots \\
w_{1}f_{1,n}+...+w_{n}f_{n,n} } \\
&= \bm{f_{1,1} & ... & f_{1,n} \\ \vdots & & \vdots \\ f_{n,1} & ... & f_{n,n}} \bm{w_{1} \\ \vdots \\ w_{n}}
\end{align*}

\begin{definicion}
 $$P_{F,B}:=\bm{f_{1,1} & ... & f_{1,n} \\ \vdots & & \vdots \\ f_{n,1} & ... & f_{n,n}}$$ se conoce como \emph{matriz
de paso} de $F$ a $B$. También decimos que es la matriz cambio de base de $F$ a $B$.
\end{definicion}

Así como podemos cambiar las coordenadas de la base $F$ a la base $B,$ podemos aplicar el mismo procedimiento para
encontrar la matriz de paso de $B$ a $F$. Sin embargo, al ser el procedimiento inverso, basta encontrar la matriz
inversa. En otras palabras.

\begin{proposicion}
\label{PBF}
 $P_{B,F}=P_{F,B}^{-1}.$
\end{proposicion}

\begin{observacion}
 El hecho de que $P_{F,B}$ sea invertible se debe a que esta formada por los vectores columna que son las coordenadas
de cada elemento de la base $F$ en términos de $B.$ Estos vectores generen todo $\R^{n}$, que es equivalente a
que la matriz $P_{F,B}$ sea invertible.
\end{observacion}

\begin{resuelto}
 \begin{enumerate}
  \item Verifique que $$F=\basis{
\bm{1\\0\\0} , \bm{0 \\ -2 \\ 0} , \bm{1 \\ 0 \\ 1}
  }$$ es una base de $\R^{3}.$
  \item Si denotamos por $E$ la base estandar de $\R^{3}$, encuentre las matrices de paso $P_{F,E}$ y $P_{E,F}$.
 \end{enumerate}

\begin{solucion}
 Por la proposición \ref{prop:1}, basta verificar que $F$ es un conjunto linealmente independiente. Ahora bien, por la
proposición \ref{prop:2}, basta verificar que
 $$\begin{vmatrix}
    1 & 0 & 1 \\ 0 & -2 & 0 \\ 0 & 0 & 1
   \end{vmatrix}\neq 0.
$$

Aunque esto lo podemos hacer \emph{a mano}, usaremos $\texttt{WxMaxima}$ para hacer dichas cuentas. Primero
introducimos la matriz, a partir de la cual calcularemos el determinante y la denotaremos por $P.$

\noindent
%%%%%%%%%%%%%%%
%%% INPUT:
\begin{minipage}{8ex}{\color{red}\bf
\begin{verbatim}
(%i1)
\end{verbatim}}
\end{minipage}
\begin{minipage}{\textwidth}{\color{blue}
\begin{verbatim}
P: matrix(
 [1,0,1],
 [0,-2,0],
 [0,0,1]
);
\end{verbatim}}
\end{minipage}
%%% OUTPUT:
\definecolor{labelcolor}{RGB}{100,0,0}
\begin{math}\displaystyle
\parbox{8ex}{\color{labelcolor}(\%o1) }
\begin{pmatrix}1 & 0 & 1\cr 0 & -2 & 0\cr 0 & 0 & 1\end{pmatrix}
\end{math}
%%%%%%%%%%%%%%%

Posteriormente, calculamos su determinante.

\noindent
%%%%%%%%%%%%%%%
%%% INPUT:
\begin{minipage}{8ex}{\color{red}\bf
\begin{verbatim}
(%i2)
\end{verbatim}}
\end{minipage}
\begin{minipage}{\textwidth}{\color{blue}
\begin{verbatim}
determinant(%);
\end{verbatim}}
\end{minipage}
%%% OUTPUT:
\definecolor{labelcolor}{RGB}{100,0,0}
\begin{math}\displaystyle
\parbox{8ex}{\color{labelcolor}(\%o2) }
-2
\end{math}
%%%%%%%%%%%%%%%

y concluimos que $F$ es una base.

Note que
$$
\bm{1\\0\\0} , \bm{0 \\ -2 \\ 0} , \bm{1 \\ 0 \\ 1}
$$
estan ya dados en terminos de la base canónica $E$ y por tanto
$$
P_{F,E}=P.
$$

Por la proposición \ref{PBF}, sabemos que $P_{E,F}=P^{-1}$ y usando nuevamente \texttt{WxMaxima}, calculamos esta
matriz inversa.


\noindent
%%%%%%%%%%%%%%%
%%% INPUT:
\begin{minipage}{8ex}{\color{red}\bf
\begin{verbatim}
(%i3)
\end{verbatim}}
\end{minipage}
\begin{minipage}{\textwidth}{\color{blue}
\begin{verbatim}
invert(P);
\end{verbatim}}
\end{minipage}
%%% OUTPUT:
\definecolor{labelcolor}{RGB}{100,0,0}
\begin{math}\displaystyle
\parbox{8ex}{\color{labelcolor}(\%o3) }
\begin{pmatrix}1 & 0 & -1\cr 0 & -\frac{1}{2} & 0\cr 0 & 0 & 1\end{pmatrix}
\end{math}.
%%%%%%%%%%%%%%%

\end{solucion}


\end{resuelto}


\subsection*{Ejemplos}

\begin{resuelto}
 \label{matriz:paso}
 Encuentre las matrices de paso $P_{F,E}$ y $P_{E,F}$ para los siguientes casos.
  \begin{enumerate}
  \item $E$ la base canónica de $\R^{2}$, $F=\left(\bm{1\\0},\bm{0\\-1}\right)$,
  \item $E$ la base canónica de $\R^{2}$, $F=\left(\bm{1\\0},\bm{1\\1}\right)$
  \item $E$ la base canónica de $\R^{3}$, $F=\left(\bm{1\\0\\0},\bm{1\\1\\0},\bm{1\\1\\1}\right)$.
 \end{enumerate}

\end{resuelto}


\begin{resuelto}
 Encuentre las coordenadas de los siguientes vectores $v$, en las bases ordendas $F$ indicadas. Utilice el resultado en
el ejercicio \ref{matriz:paso}.
 \begin{enumerate}
  \item $v=\bm{-1\\2}$, $F=\left(\bm{1\\0},\bm{0\\-1}\right)$,
  \item $v=\bm{-1\\2}$, $F=\left(\bm{1\\0},\bm{1\\1}\right)$
  \item $v=\bm{3\\-1\\5}$, $F=\left(\bm{1\\0\\0},\bm{1\\1\\0},\bm{1\\1\\1}\right)$.
 \end{enumerate}

\end{resuelto}

\begin{resuelto}
 Encuentre las coordendas de los elementos de la base canonica de $V$ en terminos de las bases ordenadas $F$ indicadas.
Utilice el resultado en el ejercicio \ref{matriz:paso}.
 \begin{enumerate}
  \item $V=\R^{2}$, $F=\left(\bm{1\\0},\bm{0\\-1}\right)$,
  \item $V=\R^{2}$, $F=\left(\bm{1\\0},\bm{1\\1}\right)$
  \item $V=\R^{3}$, $F=\left(\bm{1\\0\\0},\bm{1\\1\\0},\bm{1\\1\\1}\right)$.
 \end{enumerate}
\end{resuelto}

